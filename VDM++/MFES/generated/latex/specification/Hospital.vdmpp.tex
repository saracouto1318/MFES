\begin{vdmpp}[breaklines=true]
class Hospital
instance variables
  private medicalAssociated: set of (HealthProfessional);
  private name: Types`String;
  private address: Types`String;
  private tasks: set of(Task);
  private trainings: set of(Training);
  private safetyNet: [SafetyNetHospital];
 
 inv safetyNet <> nil; 
 inv card medicalAssociated >= 0;
 inv card tasks >= 0;
operations

(*@
\label{Hospital:15}
@*)
 public Hospital: Types`String * Types`String * SafetyNetHospital ==> Hospital
  Hospital(n, a, s) == (name := n; address := a; safetyNet := s; medicalAssociated := {}; tasks := {}; trainings := {}; 
  safetyNet.addHospital(self); return self)
 pre safetyNet <> nil
 post name = n and address = a and safetyNet = s and medicalAssociated = {} and tasks = {} and trainings = {};
(*@
\label{getName:20}
@*)
 
 pure public getName: () ==> Types`String
  getName() == (return name);
(*@
\label{getAddress:23}
@*)
 
 pure public getAddress: () ==> Types`String
  getAddress() == (return address);
(*@
\label{addMedAssociated:26}
@*)
 
 public addMedAssociated: HealthProfessional ==> ()
  addMedAssociated(d) == (medicalAssociated := {d} union medicalAssociated)
 pre d not in set medicalAssociated
 post d in set medicalAssociated;
(*@
\label{removeMedAssociated:31}
@*)
  
 public removeMedAssociated: HealthProfessional ==> ()
  removeMedAssociated(d) == (
                for all t in set tasks do
                 if(d = t.getMedAssoc())
                  then removeTask(t);
                for all t in set trainings do
                 if(d = t.(*@\vdmnotcovered{getMedAssoc}@*)())
                  then removeTraining((*@\vdmnotcovered{t}@*));
                medicalAssociated := medicalAssociated \ {d})
 pre d in set medicalAssociated
 post d not in set medicalAssociated;
(*@
\label{addTask:43}
@*)
 
 public addTask: Task ==> ()
  addTask(d) == (
          if(d.getPatient() not in set d.getMedAssoc().getPatients())
           then d.getMedAssoc().addPatient(d.getPatient());
          tasks := {d} union tasks)
 pre d not in set tasks and forall t in set tasks & 
  not (overlap(d.getSchedule(), t.getSchedule()) and d.getMedAssoc().getCC() = t.getMedAssoc().getCC()
  and d.getPatient().getCC() = t.getPatient().getCC() and d.getMedAssoc().getCC() = t.getPatient().getCC() (*@\vdmnotcovered{and}@*) 
(*@
\label{removeTask:52}
@*)
  d.getPatient().(*@\vdmnotcovered{getCC}@*)() = t.(*@\vdmnotcovered{getMedAssoc}@*)().(*@\vdmnotcovered{getCC}@*)())
 post d in set tasks and d.getPatient() in set d.getMedAssoc().getPatients();
  
 public removeTask: Task ==> ()
  removeTask(d) == (tasks := tasks \ {d})
(*@
\label{addTraining:57}
@*)
 pre d in set tasks
 post d not in set tasks;
 
 public addTraining: Training ==> ()
  addTraining(d) == (trainings := {d} union trainings)
(*@
\label{removeTraining:62}
@*)
 pre d not in set trainings and forall t in set trainings & not (overlap(d.(*@\vdmnotcovered{getSchedule}@*)(), t.(*@\vdmnotcovered{getSchedule}@*)()))
 post d in set trainings;
  
 public removeTraining: Training ==> ()
  removeTraining(d) == (trainings := trainings \ {d})
(*@
\label{getTasksByType:67}
@*)
 pre d in set trainings
 post d not in set trainings;
 
 pure public getTasksByType: Types`TaskType ==> set of (Task)
  getTasksByType(s) == (
              dcl tasksTotal: set of (Task);
              tasksTotal := {};
              for all t in set tasks do
               if(t.getType() = s)
(*@
\label{getTrainingsByType:76}
@*)
                then tasksTotal := tasksTotal union {t};
                
              return tasksTotal);
              
 pure public getTrainingsByType: Types`Purpose ==> set of (Training)
  getTrainingsByType(s) == (
              dcl train: set of (Training);
              train := {};
              for all t in set trainings do
(*@
\label{getMedicalAssociatedByType:85}
@*)
               if(t.getPurpose() = s)
                then train := train union {t};
                
              return train);

 pure public getMedicalAssociatedByType: Types`Type ==> set of (HealthProfessional)
  getMedicalAssociatedByType(type) == (
           dcl med: set of(HealthProfessional);
           med := {};
(*@
\label{overlap:94}
@*)
           for all d in set medicalAssociated do
            if(d.getType() = type)
             then med := med union {d};
            
           return med);
           
 pure public overlap: Schedule * Schedule ==> bool
  overlap(t1, t2) == (
             if(t1.compareDate(t1.getScheduleStart(), t2.getScheduleStart()) 
              or (not t1.(*@\vdmnotcovered{compareDateLess}@*)(t1.getScheduleStart(), (*@\vdmnotcovered{t2}@*).(*@\vdmnotcovered{getScheduleStart}@*)()) 
              or not t1.compareDateLess((*@\vdmnotcovered{t1}@*).(*@\vdmnotcovered{getScheduleEnd}@*)(), (*@\vdmnotcovered{t2}@*).(*@\vdmnotcovered{getScheduleStart}@*)())))
              then return true
             else
              return (*@\vdmnotcovered{false}@*));
              
end Hospital
\end{vdmpp}
\bigskip
\begin{longtable}{|l|r|r|r|}
\hline
Function or operation & Line & Coverage & Calls \\
\hline
\hline
\hyperref[Hospital:15]{Hospital} & 15&100.0\% & 154 \\
\hline
\hyperref[addMedAssociated:26]{addMedAssociated} & 26&100.0\% & 65 \\
\hline
\hyperref[addTask:43]{addTask} & 43&68.5\% & 0 \\
\hline
\hyperref[addTraining:57]{addTraining} & 57&71.4\% & 5 \\
\hline
\hyperref[getAddress:23]{getAddress} & 23&100.0\% & 39 \\
\hline
\hyperref[getMedicalAssociatedByType:85]{getMedicalAssociatedByType} & 85&100.0\% & 346 \\
\hline
\hyperref[getName:20]{getName} & 20&100.0\% & 112 \\
\hline
\hyperref[getTasksByType:67]{getTasksByType} & 67&100.0\% & 115 \\
\hline
\hyperref[getTrainingsByType:76]{getTrainingsByType} & 76&100.0\% & 20 \\
\hline
\hyperref[overlap:94]{overlap} & 94&39.2\% & 249 \\
\hline
\hyperref[removeMedAssociated:31]{removeMedAssociated} & 31&76.6\% & 0 \\
\hline
\hyperref[removeTask:52]{removeTask} & 52&100.0\% & 22 \\
\hline
\hyperref[removeTraining:62]{removeTraining} & 62&100.0\% & 5 \\
\hline
\hline
Hospital.vdmpp & & 82.4\% & 1132 \\
\hline
\end{longtable}

