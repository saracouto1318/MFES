\begin{vdmpp}[breaklines=true]
class Hospital
instance variables
  private medicalAssociated: set of (HealthProfessional);
  private name: Types`String;
  private address: Types`String;
  private tasks: set of(Task);
  private trainings: set of(Training);
  private safetyNet: [SafetyNetHospital];
 
 inv name <> [] and address <> [];
 inv safetyNet <> nil; 
 inv card medicalAssociated >= 0;
 inv card tasks >= 0;
operations

(*@
\label{Hospital:16}
@*)
 public Hospital: Types`String * Types`String * SafetyNetHospital ==> Hospital
  Hospital(n, a, s) == (name := n; address := a; safetyNet := s; medicalAssociated := {}; tasks := {}; trainings := {}; return self)
 pre n <> [] and a <> [] and safetyNet <> nil
 post name = n and address = a and safetyNet = s and medicalAssociated = {} and tasks = {} and trainings = {};
 
(*@
\label{getName:21}
@*)
 pure public getName: () ==> Types`String
  getName() == (return name);
 
(*@
\label{getAddress:24}
@*)
 pure public getAddress: () ==> Types`String
  getAddress() == (return address);
 
(*@
\label{getSafetyNet:27}
@*)
 pure public addMedAssociated: HealthProfessional ==> set of (HealthProfessional)
  addMedAssociated(d) == (return ({d} union medicalAssociated))
 pre d not in set medicalAssociated
(*@
\label{addMedAssociated:30}
@*)
 post d in set medicalAssociated;
  
 pure public removeMedAssociated: HealthProfessional ==> set of (HealthProfessional)
  removeMedAssociated(d) == (
                for all t in set tasks do
(*@
\label{removeMedAssociated:35}
@*)
                 if(d = t.getMedAssoc())
                  then removeTask(t);
                for all t in set trainings do
                 if(d = t.getMedAssoc())
                  then removeTraining(t);
(*@
\label{addTask:40}
@*)
                return (medicalAssociated \ {d}))
 pre d in set medicalAssociated
 post d not in set medicalAssociated;
 
 public addTask: Task ==> set of (Task)
  addTask(d) == (
          dcl patients : set of(Patient);
          if(d.getPatient() not in set d.getMedAssoc().getPatients())
(*@
\label{removeTask:48}
@*)
           then patients := d.getMedAssoc().addPatient(d.getPatient());
          return ({d} union tasks))
 pre d not in set tasks and forall t in set tasks & 
  not (overlap(d.getSchedule(), t.getSchedule()) and not (d.getMedAssoc().getCC() <> t.getMedAssoc().getCC() and 
    d.getPatient().getCC() <> t.getPatient().getCC() and d.getMedAssoc().getCC() <> t.getPatient().getCC()
(*@
\label{addTraining:53}
@*)
    and d.getPatient().getCC() <> t.getMedAssoc().getCC()))
 post d in set tasks and d.getPatient() in set d.getMedAssoc().getPatients();
  
 pure public removeTask: Task ==> set of (Task)
  removeTask(d) == (return (tasks \ {d}))
(*@
\label{removeTraining:58}
@*)
 pre d in set tasks
 post d not in set tasks;
 
 public addTraining: Training ==> set of (Training)
  addTraining(d) == (return ({d} union trainings))
(*@
\label{getTasksByType:63}
@*)
 pre d not in set trainings and forall t in set trainings & not (overlap(d.getSchedule(), t.getSchedule()))
 post d in set trainings;
  
 pure public removeTraining: Training ==> set of (Training)
  removeTraining(d) == (return (trainings \ {d}))
 pre d in set trainings
 post d not in set trainings;
 
 pure public getTasksByType: Types`TaskType ==> set of (Task)
(*@
\label{getTrainingsByType:72}
@*)
  getTasksByType(s) == (
              dcl tasksTotal: set of (Task);
              for all t in set tasks do
               if(t.getType() = s)
                then tasksTotal := tasksTotal union {t};
                
              return tasksTotal);
              
 pure public getTrainingsByType: Types`Purpose ==> set of (Training)
(*@
\label{getMedicalAssociatedByType:81}
@*)
  getTrainingsByType(s) == (
              dcl train: set of (Training);
              for all t in set trainings do
               if(t.getPurpose() = s)
                then train := train union {t};
                
              return train);

 pure public getMedicalAssociatedByType: Types`Type ==> set of (HealthProfessional)
(*@
\label{overlap2:90}
@*)
  getMedicalAssociatedByType(type) == (
           dcl med: set of(HealthProfessional);
           for all d in set medicalAssociated do
            if(d.getType() = type)
             then med := med union {d};
            
           return med);
           
 pure public overlap: Schedule * Schedule ==> bool
  overlap(t1, t2) == (
             if(t1.compareDate(t1.getScheduleStart(), t2.getScheduleStart()) 
(*@
\label{overlapTraining:101}
@*)
              or (t1.compareDateLess(t1.getScheduleStart(), t2.getScheduleStart()) 
              and not t1.compareDateLess(t1.getScheduleEnd(), t2.getScheduleStart()))
              or (not t1.compareDateLess(t1.getScheduleStart(), t2.getScheduleStart()) 
              and t1.compareDateLess(t1.getScheduleStart(), t2.getScheduleEnd())))
              then return true
             else
              return false);
              
end Hospital
\end{vdmpp}
